\documentclass[10pt,a4paper]{article}
\usepackage{bm}
\usepackage{amsmath}
\title{Homework 3 on ASTR400B}
\author{Jiachuan Xu}

\begin{document}
\maketitle
\section{Output Table: Mass Break Down of Local Group}
\begin{table}[h]
\begin{tabular}{cccccc}
Name & $M_{Halo}$ & $M_{Disk}$ & $M_{Bulge}$ & $M_{Total}$ & $f_{bar}$ \\
 & $\mathrm{1 \times 10^{12}\,M_{\odot}}$ & $\mathrm{1 \times 10^{12}\,M_{\odot}}$ & $\mathrm{1 \times 10^{12}\,M_{\odot}}$ & $\mathrm{1 \times 10^{12}\,M_{\odot}}$ & $\mathrm{}$ \\
M31 & 1.921 & 0.12 & 0.019 & 2.06 & 0.068 \\
M33 & 0.187 & 0.009 & 0.0 & 0.196 & 0.047 \\
MW & 1.975 & 0.075 & 0.01 & 2.06 & 0.041 \\
Galaxy Group & 4.082 & 0.204 & 0.029 & 4.316 & 0.054 \\
\end{tabular}
\caption{\label{tab:}Mass break down of local group (snapnumber: 0)}
\end{table}

\section{Answers to The Questions on Homework 3}
\begin{itemize}
	\item The total mass of MW and M31 are the same in this simulation, the mass is dominated by dark matter halo, which composes $93.25\%$ of M31 and $95.87\%$ of MW.
	\item The stellar mass of M31 is more than MW on both disk and bulge components, and the baryon fraction of M31 is $65.85\%$ higher than the MW's. This indicates that M31 should have a higher luminosity, providing these two galaxy have similiar stellar population.
	\item The dark matter of MW is only $2.81\%$ more than M31's. Given the dominating fraction the dark matter takes, the little difference is understandable. The difference in stellar mass may be a result of different assembly history.
	\item The baryon fraction of M31, M33 and MW are $6.8\%$, $4.7\%$ and $4.1\%$, which is much lower than the average fraction $16\%$. If the mass of gas in the disk in negligible compared to the stellar mass, the "missing baryons" may hide in non-luminous IGM, low-luminosity dwarfs or white dwarfs
\end{itemize}
\end{document}